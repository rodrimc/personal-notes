\Ceu\ compiler enforces the following additional restrition not stated in     
the semantics: if $p=\ceu{\Loop{p'}}$, then either $\blocked(p')=1$ or
$p'$ has a reachable $\ceu\Break$ statement.
\todo{Não sei se ficou bom esse texto, especialmente pelo fato de computar 
se um trecho de um programa é ``reachable'' ser um problema indecidível =p.}

\begin{proof}
  By induction on the structure of programs.
  \begin{case}  
  \item$p=\ceu{\Await(e)}$, for some~$e\in{E}$.  By~\Cref{def:orig:blocked}
    $\blocked(\ceu{\Await(e)},\alpha,m) = 0$, therefore
    by~\Cref{def:orig:outer-step} and by axiom~\eqref{def:orig:inner:await}, 
    $\delta=<\ceu{\Awaiting(e,n')},\alpha,m>$, with~$n'=n+1$.
    %%
  \item$p=\ceu{\Awaiting(e,n')}$, for some~$e\in{E}$ and~$n'\in{N}$.
    \todo{Precisa revisar esse caso}
    \begin{case}
      \item $e\ne \alpha_{[1]}$ or $n'\ne \|\alpha|$. By \Cref{def:orig:blocked},
        $\blocked(\ceu{\Awaiting(e,n')},\alpha,m)=1$, therefore,
        by~\Cref{def:orig:outer-step}, 
        $\delta=<\ceu{\Awaiting(e,n')},\alpha_{[2]}\ldots\alpha_{[n]},m>$.
        %%
      \item otherwise, $\blocked(\ceu{\Awaiting(e,n')},\alpha,m)=0$.
        Then there are two cases. Either $n'\leq n$ or $n'>n$.
        If $n'\leq n$, then by~\Cref{def:orig:outer-step} and by
        axiom~\eqref{def:orig:inner:awaiting}, $\delta=\<\ceu\Skip,\alpha,m>$.
        Otherwise, if $n'>n$, then ...\todo{Aqui o teorema falha, pois
          $\blocked(p...)=0$ e não existe $\delta$}.
    \end{case}
    %%
  \item$p=\ceu{\Emit(e)}$, for some~$e\in{E}$.  By \Cref{def:orig:blocked},
    $\blocked(\ceu{\Emit(e)},\alpha,m) = 0$, therefore
    by~\Cref{def:orig:outer-step} and by axiom~\eqref{def:orig:inner:emit}, 
    $\delta=\<\ceu{\Emitting(n')},e\alpha,m>$, with~$n'=\|\alpha|$.
    %%
  \item$p=\ceu{\Emitting(e,n')}$, for some~$e\in{E}$ and~$n'\in{N}$.
    \begin{case}
      \item $n = \|\alpha|$. By \Cref{def:orig:blocked},
      $\blocked(\ceu{\Emitting(e,n')},\alpha,m) = 0$, therefore
      by~\Cref{def:orig:outer-step} and by axiom~\eqref{def:orig:inner:emitting}, 
      $\delta=\<\ceu\Skip,\alpha,m>$.
      %%
      \item $n'\ne \|\alpha|$. By \Cref{def:orig:blocked},
      $\blocked(\ceu{\Emitting(e,n')},\alpha,m) = 1$, therefore
      by~\Cref{def:orig:outer-step}
      $\delta=\<\ceu{Emitting(e,n')},\alpha_{[2]}...\alpha_{[n]},m>$.
    \end{case}
    %%
  \item$p=\ceu{\Ifelse{b}{p'}{p''}}$, for some~$b\in{B}$ and~$p'$,
    $p''\in{P}$.
    \begin{case}
    \item$\eval(b,m)=1$.  By \Cref{def:orig:blocked},
      $\blocked(\ceu{\Ifelse{b}{p'}{p''}},\alpha,m) = 0$, therefore
      by~\Cref{def:orig:outer-step} and by axiom~\eqref{def:orig:inner:if-true}, 
      $\delta=\<\ceu{p'},\alpha,m>$.
    %%
    \item$\eval(b,m)=0$.  By \Cref{def:orig:blocked},
      $\blocked(\ceu{\Ifelse{b}{p'}{p''}},\alpha,m) = 0$, therefore
      by~\Cref{def:orig:outer-step} and by axiom~\eqref{def:orig:inner:if-true}, 
      $\delta=\<\ceu{p''},\alpha,m>$.
    \end{case}
    %%
  \item$p=\ceu{p';p''}$, for some~$p'$, $p''\in{P}$. 
    By \Cref{def:orig:blocked},
    $\blocked(\ceu{p';p''},\alpha,m)=\blocked(\ceu{p'},\alpha,m)$.

    \begin{case}
      \item$p'=\ceu\Skip$.  By \Cref{def:orig:blocked},
        $\blocked(\ceu\Skip,\alpha,m)=0$, therefore
        by~\Cref{def:orig:outer-step} and by axiom~\eqref{def:orig:inner:seq-skip}, 
        $\delta=\<\ceu{p''},\alpha,m>$.
      %%
      \item$p'=\ceu{\Attr{v}{a}}$, for some $v\in{V}$ and~$a\in{A}$.
        By \Cref{def:orig:blocked}, $\blocked(\ceu{\Attr{v}{a}},\alpha,m)=0$,
        therefore by~\Cref{def:orig:outer-step} and by
        axiom~\eqref{def:orig:inner:seq-attr}, 
        $\delta=\<\ceu{p''},\alpha,m[v/\eval(a)]>$.
      %%
      \item$p'=\ceu{\Break}$. By \Cref{def:orig:blocked},
        $\blocked(\ceu\Break,\alpha,m)=0$, therefore
        by~\Cref{def:orig:outer-step} and by
        axiom~\eqref{def:orig:inner:seq-break}, 
        $\delta=\<\ceu\Break,\alpha,m>$.
      %%
      \item$p'\ne\ceu{\Skip,\Attr{v}{a}},\ceu{\Break}$.
        By~\Cref{lem:orig:inner-step-or-blocked}, $p'$ either has an applicable
        rule or $\blocked(\ceu{p'},\alpha,m)=1$.

        If $\blocked(\ceu{p'},\alpha,m)=0$, then by
        rule~\eqref{def:orig:inner:seq}, there is a $p_1'$ such as 
        \begin{align*}
          \<p',\alpha,m>\step{n}\<p_1',\alpha',m'>
        \end{align*}
        for some~$p_1'\in{P}$, $\alpha'\in{E}$ and~$m'\in\mathcal{M}$.
        Thus, by~\Cref{def:orig:outer-step}, $\delta=\<p_1',\alpha',m'>$.

        If, otherwise, $\blocked(\ceu{p'},\alpha,m)=1$, then
        by~\Cref{def:orig:outer-step},
        $\delta=\<\ceu{Emitting(e,n')},\alpha_{[2]}...\alpha_{[n]},m>$.
    \end{case}
  %   %%
  \item$p=\ceu{\Loop{p'}}$.   By \Cref{def:orig:blocked},
      $\blocked(\ceu{\Loop{p'}},\alpha,m) = 0$, therefore
      by~\Cref{def:orig:outer-step} and by axiom~\eqref{def:orig:inner:loop}, 
      $\delta=\<\ceu{p'\Atloop{p'}},\alpha,m>$.
    %%
  \item$p=\ceu{p'\Atloop{p''}}$.
    By \Cref{def:orig:blocked},
    $\blocked(\ceu{p'\Atloop{p''}},\alpha,m)=\blocked(\ceu{p'},\alpha,m)$.
    \begin{case}
      \item$p'=\ceu\Skip$.  By \Cref{def:orig:blocked},
      $\blocked(\ceu\Skip,\alpha,m) = 0$, therefore
      by~\Cref{def:orig:outer-step} and by
      axiom~\eqref{def:orig:inner:atloop-skip},
      $\delta=\<\ceu{\Loop{p'}},\alpha,m>$.
      %%
      \item$p'=\ceu{\Attr{v}{a}}$, for some~$a\in{A}$, $v\in{V}$. 
      By \Cref{def:orig:blocked},
      $\blocked(\ceu{\Attr{v}{a}},\alpha,m) = 0$, therefore
      by~\Cref{def:orig:outer-step} and by
      axiom~\eqref{def:orig:inner:atloop-attr},
      $\delta=\<\ceu{\Loop{p'}},\alpha,m[v/\eval(a)]>$.
      %%
      \item$p'=\ceu\Break$.  By \Cref{def:orig:blocked},
      $\blocked(\ceu\Break,\alpha,m) = 0$, therefore
      by~\Cref{def:orig:outer-step} and by
      axiom~\eqref{def:orig:inner:atloop-break},
      $\delta=\<\ceu\Skip,\alpha,m>$.
      %%
      \item $p'\ne\ceu{\Skip,\Attr{v}{a}},\ceu{\Break}$ and.  
        By~\Cref{lem:orig:inner-step-or-blocked}, $p'$ either has an applicable
        rule or $\blocked(\ceu{p'},\alpha,m)=1$.

        If $\blocked(\ceu{p'},\alpha,m)=0$, then by
        rule~\eqref{def:orig:inner:atloop}, there is a $p_1'$ such as 
        \begin{align*}
          \<p',\alpha,m>\step{n}\<p_1',\alpha',m'>
        \end{align*}
        for some~$p_1'\in{P}, \alpha'\in{E}$ and $m'\in\mathcal{M}$.  
        Thus, by~\Cref{def:orig:outer-step}, $\delta=\<p_1',\alpha',m'>$.
        
        If, otherwise, $\blocked(\ceu{p'},\alpha,m)=1$, then
        by~\Cref{def:orig:outer-step},
        $\delta=\<\ceu{p'\Atloop{p'}},\alpha_{[2]}...\alpha_{[n]},m>$.
    \end{case}
  %   %%
  \item$p=\ceu{p'\And{p''}}$. By \Cref{def:orig:blocked},
    $\blocked(\ceu{p'\And{p''}},\alpha,m)=\blocked(\ceu{p'},\alpha,m)\cdot\blocked(\ceu{p''},\alpha,m)$.
    \begin{case}
      \item$p'=\ceu\Skip$. By~\Cref{def:orig:blocked},
        $\blocked(\ceu\Skip,\alpha,m) = 0$, therefore
        by~\Cref{def:orig:outer-step} and by
        axiom~\eqref{def:orig:inner:and-skip-left},
        $\delta=\<\ceu{p'},\alpha,m>$.
        %%
      \item$p''=\ceu\Skip$. 
        There are two cases. Either $\blocked(\ceu{p'},\alpha,m)=0$
        or $\blocked(\ceu{p'},\alpha,m)=1$. If $\blocked(\ceu{p'},\alpha,m)=0$,
        then the derivation of~$\delta$ is similar
        to~\Cref{thm:orig:term-outer:and}. 
        Otherwise, if $\blocked(\ceu{p'},\alpha,m)=1$, then,
        as by~\Cref{def:orig:blocked} $\blocked(\ceu\Skip,\alpha,m) = 0$,
        by~\Cref{def:orig:outer-step} and by
        axiom~\eqref{def:orig:inner:and-skip-right},
        $\delta=\<\ceu{p'},\alpha,m>$.
        %%
      \item$p'=\ceu{\Attr{v}{a}}$, for some~$v\in{V}$ and $a\in{A}$.
        By~\Cref{def:orig:blocked}, $\blocked(\ceu{\Attr{v}{a}},\alpha,m) = 0$,
        therefore by~\Cref{def:orig:outer-step} and by
        axiom~\eqref{def:orig:inner:and-attr-left},
        $\delta=\<\ceu{p''},\alpha,m[v/\eval(a)]>$.
        %%
      \item$p''=\ceu{\Attr{v}{a}}$, for some~$v\in{V}$, $a\in{A}$.
        There are two cases. Either $\blocked(\ceu{p'},\alpha,m)=0$
        or $\blocked(\ceu{p'},\alpha,m)=1$. If $\blocked(\ceu{p'},\alpha,m)=0$,
        then the derivation of~$\delta$ is similar
        to~\Cref{thm:orig:term-outer:and}. 
        Otherwise, if $\blocked(\ceu{p'},\alpha,m)=1$, then,
        as by~\Cref{def:orig:blocked} 
        $\blocked(\ceu{\Attr{v}{a}},\alpha,m) = 0$,
        by~\Cref{def:orig:outer-step} and by
        axiom~\eqref{def:orig:inner:and-attr-right},
        $\delta=\<\ceu{p'},\alpha,m[v/\eval(a)]>$.
        %%
      \item$p'=\ceu\Break$. By~\Cref{def:orig:blocked},
        $\blocked(\ceu\Break,\alpha,m) = 0$, therefore
        by~\Cref{def:orig:outer-step} and by
        axiom~\eqref{def:orig:inner:and-break-left},
        $\delta=\<\ceu{\clear(p'');\Break},\alpha,m>$.
        %%
      \item$p''=\ceu\Break$.
        There are two cases. Either $\blocked(\ceu{p'},\alpha,m)=0$
        or $\blocked(\ceu{p'},\alpha,m)=1$. If $\blocked(\ceu{p'},\alpha,m)=0$,
        then the derivation of~$\delta$ is similar
        to~\Cref{thm:orig:term-outer:and}. 
        Otherwise, if $\blocked(\ceu{p'},\alpha,m)=1$, then,
        as by~\Cref{def:orig:blocked} $\blocked(\ceu\Break,\alpha,m) = 0$,
        by~\Cref{def:orig:outer-step} and by
        axiom~\eqref{def:orig:inner:and-break-right},
        $\delta=\<\ceu{\clear(p');\Break},\alpha,m>$.
        %%
    \item\label{thm:orig:term-outer:and}
        $p'\ne\ceu{\Skip,\Attr{v}{a}},\ceu{\Break}$ and 
        $p''\ne\ceu{\Skip,\Attr{v}{a}},\ceu{\Break}$.  

        By~\Cref{lem:orig:inner-step-or-blocked}, $p'$ either has an applicable
        rule or $\blocked(\ceu{p'},\alpha,m)=1$.

        If $\blocked(\ceu{p'},\alpha,m)=0$, then by
        rule~\eqref{def:orig:inner:and-left}, there is a $p_1'$ such as 
        \begin{align*}
          \<p',\alpha,m>\step{n}\<p_1',\alpha',m'>
        \end{align*}
        for some~$p_1'\in{P}, \alpha'\in{E}$ and $m'\in\mathcal{M}$.  
        Thus, by~\Cref{def:orig:outer-step}, $\delta=\<p_1',\alpha',m'>$.
        
        If, otherwise, $\blocked(\ceu{p'},\alpha,m)=1$, then there
        are two cases. Either~$\blocked(\ceu{p''},\alpha,m)=0$ 
        or~$\blocked(\ceu{p''},\alpha,m)=1$. 

        If $\blocked(\ceu{p''},\alpha,m)=0$, then by
        rule~\eqref{def:orig:inner:and-right}, there is a $p_1''$ such as 
        \begin{align*}
          \<p'',\alpha,m>\step{n}\<p_1'',\alpha'',m''>
        \end{align*}
        for some~$p_1''\in{P}, \alpha''\in{E}$ and $m''\in\mathcal{M}$.  
        Thus, by~\Cref{def:orig:outer-step}, $\delta=\<p_1'',\alpha'',m''>$.

        If, otherwise, $\blocked(\ceu{p''},\alpha,m)=1$, then
        by~\Cref{def:orig:outer-step},
        $\delta=\<\ceu{p'\And{p'}},\alpha_{[2]}...\alpha_{[n]},m>$.
    \end{case}
    %%
  \item$p=\ceu{p'\Or{p''}}$. By \Cref{def:orig:blocked},
    $\blocked(\ceu{p'\Or{p''}},\alpha,m)=\blocked(\ceu{p'},\alpha,m)\cdot\blocked(\ceu{p''},\alpha,m)$.
    \begin{case}
      \item$p'=\ceu\Skip$. By~\Cref{def:orig:blocked},
        $\blocked(\ceu\Skip,\alpha,m) = 0$, therefore
        by~\Cref{def:orig:outer-step} and by
        axiom~\eqref{def:orig:inner:or-skip-left},
        $\delta=\<\ceu{\clear(p')},\alpha,m>$.
        %%
      \item$p''=\ceu\Skip$. 
        There are two cases. Either $\blocked(\ceu{p'},\alpha,m)=0$
        or $\blocked(\ceu{p'},\alpha,m)=1$. If $\blocked(\ceu{p'},\alpha,m)=0$,
        then the derivation of~$\delta$ is similar
        to~\Cref{thm:orig:term-outer:or}. 
        Otherwise, if $\blocked(\ceu{p'},\alpha,m)=1$, then,
        as by~\Cref{def:orig:blocked} $\blocked(\ceu\Skip,\alpha,m) = 0$,
        by~\Cref{def:orig:outer-step} and by
        axiom~\eqref{def:orig:inner:or-skip-right},
        $\delta=\<\ceu{\clear(p')},\alpha,m>$.

        %%
      \item$p'=\ceu{\Attr{v}{a}}$, for some~$v\in{V}$ and $a\in{A}$.
        By~\Cref{def:orig:blocked}, $\blocked(\ceu{\Attr{v}{a}},\alpha,m) = 0$,
        therefore by~\Cref{def:orig:outer-step} and by
        axiom~\eqref{def:orig:inner:or-attr-left},
        $\delta=\<\ceu{\clear(p'')},\alpha,m[v/\eval(a)]>$.
        %%
      \item$p''=\ceu{\Attr{v}{a}}$, for some~$v\in{V}$, $a\in{A}$.
        There are two cases. Either $\blocked(\ceu{p'},\alpha,m)=0$
        or $\blocked(\ceu{p'},\alpha,m)=1$. If $\blocked(\ceu{p'},\alpha,m)=0$,
        then the derivation of~$\delta$ is similar
        to~\Cref{thm:orig:term-outer:or}. 
        Otherwise, if $\blocked(\ceu{p'},\alpha,m)=1$, then,
        as by~\Cref{def:orig:blocked} $\blocked(\ceu{\Attr{v}{a}},\alpha,m) = 0$,
        by~\Cref{def:orig:outer-step} and by
        axiom~\eqref{def:orig:inner:or-attr-right},
        $\delta=\<\ceu{\clear(p')},\alpha,m[v/\eval(a)]>$.
        %%
      \item$p'=\ceu\Break$. By~\Cref{def:orig:blocked},
        $\blocked(\ceu\Break,\alpha,m) = 0$, therefore
        by~\Cref{def:orig:outer-step} and by
        axiom~\eqref{def:orig:inner:or-break-left},
        $\delta=\<\ceu{\clear(p'');\Break},\alpha,m>$.
        %%
      \item$p''=\ceu\Break$.
        There are two cases. Either $\blocked(\ceu{p'},\alpha,m)=0$
        or $\blocked(\ceu{p'},\alpha,m)=1$. If $\blocked(\ceu{p'},\alpha,m)=0$,
        then the derivation of~$\delta$ is similar
        to~\Cref{thm:orig:term-outer:or}. 
        Otherwise, if $\blocked(\ceu{p'},\alpha,m)=1$, then,
        as by~\Cref{def:orig:blocked} $\blocked(\ceu\Break,\alpha,m) = 0$,
        by~\Cref{def:orig:outer-step} and 
        axiom~\eqref{def:orig:inner:or-break-right},
        $\delta=\<\ceu{\clear(p');\Break},\alpha,m>$.
        %%
    \item\label{thm:orig:term-outer:or}
        $p'\ne\ceu{\Skip,\Attr{v}{a}},\ceu{\Break}$ and 
        $p''\ne\ceu{\Skip,\Attr{v}{a}},\ceu{\Break}$.  

        By~\Cref{lem:orig:inner-step-or-blocked}, $p'$ either has an applicable
        rule or $\blocked(\ceu{p'},\alpha,m)=1$.

        If $\blocked(\ceu{p'},\alpha,m)=0$, then by
        rule~\eqref{def:orig:inner:or-left}, there is a $p_1'$ such as 
        \begin{align*}
          \<p',\alpha,m>\step{n}\<p_1',\alpha',m'>
        \end{align*}
        for some~$p_1'\in{P}, \alpha'\in{E}$ and $m'\in\mathcal{M}$.  
        Thus, by~\Cref{def:orig:outer-step}, $\delta=\<p_1',\alpha',m'>$.
        
        If, otherwise, $\blocked(\ceu{p'},\alpha,m)=1$, then there
        are two cases. Either~$\blocked(\ceu{p''},\alpha,m)=0$ 
        or~$\blocked(\ceu{p''},\alpha,m)=1$. 

        If $\blocked(\ceu{p''},\alpha,m)=0$, then by
        rule~\eqref{def:orig:inner:or-right}, there is a $p_1''$ such as 
        \begin{align*}
          \<p'',\alpha,m>\step{n}\<p_1'',\alpha'',m''>
        \end{align*}
        for some~$p_1''\in{P}, \alpha''\in{E}$ and $m''\in\mathcal{M}$.  
        Thus, by~\Cref{def:orig:outer-step}, $\delta=\<p_1'',\alpha'',m''>$.

        If, otherwise, $\blocked(\ceu{p''},\alpha,m)=1$, then
        by~\Cref{def:orig:outer-step},
        $\delta=\<\ceu{p'\Or{p'}},\alpha_{[2]}...\alpha_{[n]},m>$.
    \end{case}
  \end{case}
\end{proof}

