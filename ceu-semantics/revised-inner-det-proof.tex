\begin{proof}
  By induction on the structure of derivations.
  Suppose
  \[
    d_1\Vdash\<p,\alpha,m>\step{n}\<p_1,\alpha_1,m_1>
    \quad\text{and}\quad
    d_2\Vdash\<p,\alpha,m>\step{n}\<p_2,\alpha_2,m_2>,
  \]
  for some derivations~$d_1$ and~$d_2$.  Then there are eleven possibilities
  depending on the structure of~$p$.  (Note that the implication is
  trivially true for~$p=\ceu{\Break}$, as there are no rules to evaluate 
  such program.)
  \begin{case}
  \item$p=\ceu\Skip$.  Then~$d_1$ and~$d_2$ are instances of
    axiom~\eqref{def:orig:inner:skip}, and as such,
    $p_1=p_2=\ceu\Term$, and $\alpha_1=\alpha_2=\alpha$ and~$m_1=m_2=m$.
    %%
  \item$p=\ceu{\Attr{v}{a}}$.  Then~$d_1$ and~$d_2$ are instances of
    axiom~\eqref{def:orig:inner:attr}, and as such,
    $p_1=p_2=\ceu\Skip$, and $\alpha_1=\alpha_2=\alpha$
    and~$m_1=m_2=m[v/\eval(a)]$.
    %%
  \item$p=\ceu{\Await(e)}$.  Then~$d_1$ and~$d_2$ are instances of
    axiom~\eqref{def:orig:inner:await}, and as such,
    $p_1=p_2=\ceu{\Awaiting(e,n')}$ with~$n'=n+1$, and
    $\alpha_1=\alpha_2=\alpha$ and~$m_1=m_2=m$.
    %%
  \item$p=\ceu{\Awaiting(e,n')}$.  Then~$d_1$ and~$d_2$ are instances of
    axiom~\eqref{def:orig:inner:awaiting}, with~$n'\le{n}$.
    Thus~$p_1=p_2=\ceu{\Skip}$, $\alpha_1=\alpha_2=\alpha$, and~$m_1=m_2=m$.
    %%
  \item$p=\ceu{\Emit(e)}$.  Then~$d_1$ and~$d_2$ are instances of
    axiom~\eqref{def:orig:inner:emit}, and as such,
    $p_1=p2=\ceu{\Emitting(n')}$ with~$n'=\|\alpha|$,
    and~$\alpha_1=\alpha_2=e\alpha$ and~$m_1=m_2=m$.
    %%
  \item$p=\ceu{\Emitting(e,n')}$.  Then~$d_1$ and~$d_2$ are instances of
    axiom~\eqref{def:orig:inner:emitting} with~$n'=\|\alpha|$.
    Thus~$p_1=p_2=\ceu{\Skip}$, $\alpha_1=\alpha_2=\alpha$ and~$m_1=m_2=m$.
    %%
  \item$p=\ceu{\Ifelse{b}{p'}{p''}}$.
    \begin{case}
    \item$\eval(b,m)=1$.  Then~$d_1$ and~$d_2$ are instances of
      axiom~\eqref{def:orig:inner:if-true}, and as such, $p_1=p_2=p'$,
      $\alpha_1=\alpha_2=\alpha$, and~$m_1=m_2=m$.
      %%
    \item$\eval(b,m)=0$.  Then~$d_1$ and~$d_2$ are instances of
      axiom~\eqref{def:orig:inner:if-false}, and as such, $p_1=p_2=p''$,
      $\alpha_1=\alpha_2=\alpha$, and~$m_1=m_2=m$.
    \end{case}
      %%
  \item$p=\ceu{p';p''}$.
    \begin{case}
    \item$p'=\ceu\Skip$.  Then~$d_1$ and~$d_2$ are instances of
      axiom~\eqref{def:orig:inner:seq-skip}, and as such, $p_1=p_2=p''$,
      $\alpha_1=\alpha_2=\alpha$ and $m_1=m_2=m$.
      %%
    % \item$p'=\ceu{\Attr{v}{a}}$.  Then~$d_1$ and~$d_2$ are instances of
    %   axiom~\eqref{def:orig:inner:seq-attr}, and as such, $p_1=p_2=p''$,
    %   $\alpha_1=\alpha_2=\alpha$ and, as~$\eval$ is a total function,
    %   $m_1=m_2=m[v/\eval(a)]$.
      %%
    \item$p'=\ceu{\Break}$.  Then~$d_1$ and~$d_2$ are instances of
      axiom~\eqref{def:orig:inner:seq-break}, and as such, $p_1=p_2=p'$,
      $\alpha_1=\alpha_2=\alpha$ and~$m_1=m_2=m$.
      %%
    \item$p'\ne\ceu{\Skip,\Break}$.  Then~$d_1$ and~$d_2$
      are instances of rule~\eqref{def:orig:inner:seq}.  Thus there are
      derivations~$d_1'$ and~$d_2'$ such that
      \begin{align*}
        d_1'\Vdash\<p',\alpha,m>\step{n}\<p_1',\alpha_1,m_1>
        \quad\text{and}\quad
        d_2'\Vdash\<p',\alpha,m>\step{n}\<p_2',\alpha_2,m_2>,
      \end{align*}
      for some~$p_1'$, $p_2'\in{P}$.  Since~$d_1'\prec{d_1}$
      and~$d_2'\prec{d_2}$, by induction hypothesis, $\alpha_1=\alpha_2$,
      $m_1=m_2$, and~$p_1'=p_2'$, which implies
      \[
        p_1=p_1';p''=p_2';p''=p_2.
      \]
    \end{case}
    %%
  \item$p=\ceu{\Loop{p'}}$.  Then~$d_1$ and~$d_2$ are instances of
    axiom~\eqref{def:orig:inner:loop}, and as such,
    $p_1=p_2=\ceu{p'\Atloop{p'}}$, $\alpha_1=\alpha_2=\alpha$,
    and~$m_1=m_2=m$.
    %%
  \item$p=\ceu{p'\Atloop{p''}}$.
    \begin{case}
    \item$p=\ceu{\Skip\Atloop{p'}}$.  Then~$d_1$ and~$d_2$ are instances of
      axiom~\eqref{def:orig:inner:atloop-skip}, and as such,
      $p_1=p_2=\ceu{\Loop{p'}}$, $\alpha_1=\alpha_2=\alpha$,
      and~$m_1=m_2=m$.
      %%
    % \item$p=\ceu{\Attr{v}{a}\Atloop{p'}}$.  Then~$d_1$ and~$d_2$ are
    %   instances of axiom~\eqref{def:orig:inner:atloop-attr}, and as such,
    %   $p_1=p_2=\ceu{\Loop{p'}}$, $\alpha_1=\alpha_2=\alpha$, and as~$\eval$
    %   is a total function, $m_1=m_2=m[v/\eval(a)]$.
      %%
    \item$p=\ceu{\Break\Atloop{p'}}$.  Then~$d_1$ and~$d_2$ are instances of
      axiom~\eqref{def:orig:inner:atloop-break}, and as such,
      $p_1=p_2=\ceu{\Skip}$, $\alpha_1=\alpha_2=\alpha$, and~$m_1=m_2=m$.
      %%
    \item$p=\ceu{p'\Atloop{p''}}$
      with~$p'\ne\ceu{\Skip,\Break}$.  Then~$d_1$
      and~$d_2$ are instances of rule~\eqref{def:orig:inner:atloop}.  Thus
      there are derivations~$d_1'$ and~$d_2'$ such that
      \[
        d_1'\Vdash\<p',\alpha,m>\step{n}\<p_1',\alpha_1,m_1>
        \quad\text{and}\quad
        d_2'\Vdash\<p',\alpha,m>\step{n}\<p_2',\alpha_2,m_2>,
      \]
      for some~$p_1'$, $p_2'\in{P}$.  Since~$d_1'\prec{d_1}$
      and~$d_2'\prec{d_2}$, by induction hypothesis, $\alpha_1=\alpha$,
      $m_1=m_2$, and~$p_1'=p_2'$, which implies
      \[
        p_1=\ceu{p_1'\Atloop{p''}}=\ceu{p_2'\Atloop{p''}}=p_2.
      \]
    \end{case}
    %%
  \item\label{thm:orig:det-inner:and} $p=\ceu{p'\And{p''}}$.
    \begin{case}
    \item\label{thm:orig:det-inner:and-skip-left} $p'=\ceu\Skip$. Then~$d_1$
      and~$d_2$ are instances of axiom~\eqref{def:orig:inner:and-skip-left},
      and as such, $p_1=p_2=p''$, $\alpha_1=\alpha_2=\alpha$, and
      $m_1=m_2=m$.
      %%
    \item$p'\ne\ceu{\Skip,\Break}$ and~$p''=\ceu\Skip$.
      If~$\blocked(p',\alpha,m)=0$, this case
      becomes~\Cref{thm:orig:det-inner:and-left}.  Otherwise,
      if~$\blocked(p',\alpha,m)=1$, $d_1$ and~$d_2$ are instances of
      axiom~\eqref{def:orig:inner:and-skip-right}, and as such,
      $p_1=p_2=p'$, $\alpha_1=\alpha_2=\alpha$, and $m_1=m_2=m$.
      %%
    % \item\label{thm:orig:det-inner:and-attr-left} $p'=\ceu{\Attr{v}{a}}$.
    %   Then~$d_1$ and~$d_2$ are instances of
    %   axiom~\eqref{def:orig:inner:and-attr-left}, and as such,
    %   $p_1=p_2=p''$, $\alpha_1=\alpha_2=\alpha$, and as~$\eval$ is a total
    %   function, $m_1=m_2=m[v/\eval(a)]$.
      %%
    % \item $p'\ne\ceu{\Skip,\Attr{v}{a},\Break}$ and~$p''=\ceu{\Attr{v}{a}}$.
    %   If~$\blocked(p',\alpha,m)=0$, this case
    %   becomes~\Cref{thm:orig:det-inner:and-left}.  Otherwise,
    %   if~$\blocked(p',\alpha,m)=1$, then~$d_1$ and~$d_2$ are instances of
    %   axiom~\eqref{def:orig:inner:and-attr-right}, and as such,
    %   $p_1=p_2=p'$, $\alpha_1=\alpha_2=\alpha$, and as~$\eval$ is a total
    %   function, $m_1=m_2=m[v/\eval(a)]$.
      %%
    \item $p'=\ceu\Break$.  Then~$d_1$ and~$d_2$ are instances of
      axiom~\eqref{def:orig:inner:and-break-left}, and as such,
      $\alpha_1=\alpha_2=\alpha$, $m_1=m_2=m$, and as~$\clear$ is a total
      function, $p_1=p_2=\ceu{\clear(p'');\Break}$.
      %%
    \item $p'\ne\ceu{\Skip,\Break}$ and~$p''=\ceu\Break$.
      If~$\blocked(p',\alpha,m)=0$, this case becomes
      \Cref{thm:orig:det-inner:and-left}.  Otherwise,
      if~$\blocked(p',\alpha,m)=1$, then~$d_1$ and~$d_2$ are instances of
      axiom~\eqref{def:orig:inner:and-break-right}, and as such,
      $\alpha_1=\alpha_2=\alpha$, $m_1=m_2=m$, and as~$\clear$ is a total
      function, $p_1=p_2=\ceu{\clear(p');\Break}$.
      %%
    \item\label{thm:orig:det-inner:and-left}
      $p',p''\ne\ceu\Skip,\ceu\Break$.
      If~$\blocked(p',\alpha,m)=0$, $d_1$~and~$d_2$ are instances
      of~\eqref{def:orig:inner:and-left}.  Thus there are derivations~$d_1'$
      and~$d_2'$ such that
      \[
        d_1'\Vdash\<p',\alpha,m>\step{n}\<p_1',\alpha_1,m_1>
        \quad\text{and}\quad
        d_2'\Vdash\<p',\alpha,m>\step{n}\<p_2',\alpha_2,m_2>,
      \]
      for some~$p_1'$, $p_2'\in{P}$.  Since~$d_1'\prec{d_1}$
      and~$d_2'\prec{d_2}$, by induction hypothesis, $\alpha_1=\alpha_2$,
      $m_1=m_2$, and~$p_1'=p_2'$, which implies
      \[
        p_1=(\ceu{p_1'\And{p''}})=(\ceu{p_2'\And{p''}})=p_2.
      \]

      If, however, $\blocked(p',\alpha,m)=1$, $d_1$~and~$d_2$ are instances
      of~~\eqref{def:orig:inner:and-right}.  Thus there are
      derivations~$d_1''$ and~$d_2''$ such that
      \[
        d_1''\Vdash\<p'',\alpha,m>\step{n}\<p_1'',\alpha_1,m_1>
        \quad\text{and}\quad
        d_2''\Vdash\<p'',\alpha,m>\step{n}\<p_2'',\alpha_2,m_2>,
      \]
      for some~$p_1''$, $p_2''\in{P}$.  Since~$d_1''\prec{d_1}$
      and~$d_2''\prec{d_2}$, by induction hypothesis, $\alpha_1=\alpha_2$,
      $m_1=m_2$, and~$p_1''=p_2''$, which implies
      \[
        p_1=(\ceu{p'\And{p_1''}})=(\ceu{p'\And{p_2''}})=p_2.
      \]
    \end{case}
    %%
  \item$p=\ceu{p'\Or{p''}}$.
    \begin{case}
    \item$p'=\ceu\Skip$.  Then~$d_1$ and~$d_2$ are instances of
      axiom~\eqref{def:orig:inner:or-skip-left}, and as such,
      $\alpha_1=\alpha_2=\alpha$, $m_1=m_2=m$, and as~$\clear$ is a total,
      $p_1=p_2=\ceu\clear(p'')$.
      %%
    \item $p'\ne\ceu{\Skip,\Break}$ and~$p''=\ceu\Skip$.
      If~$\blocked(p',\alpha,m)=0$, this case becomes
      \Cref{thm:orig:det-inner:or-left}.  Otherwise,
      if~$\blocked(p',\alpha,m)=1$, then~$d_1$ and~$d_2$ are instances of
      axiom~\eqref{def:orig:inner:or-skip-right}, and as such,
      $\alpha_1=\alpha_2=\alpha$, $m_1=m_2=m$, and as~$\clear$ is a total
      function, $p_1=p_2=\ceu\clear(p')$.
      %%
    % \item\label{thm:orig:det-inner:or-attr-left} $p'=\ceu{\Attr{v}{a}}$.
    %   Then~$d_1$ and~$d_2$ are instances of
    %   axiom~\eqref{def:orig:inner:or-attr-left}, and as such,
    %   $\alpha_1=\alpha_2=\alpha$, and as~$\eval$ and~$\clear$ are total
    %   functions, $m_1=m_2=m[v/\eval(a)]$ and~$p_1=p_2=\ceu\clear(p'')$.
      %%
    % \item $p'\ne\ceu{\Skip,\Attr{v}{a},\Break}$ and~$p''=\ceu{\Attr{v}{a}}$.
    %   If~$\blocked(p',\alpha,m)=0$, this case becomes
    %   \Cref{thm:orig:det-inner:or-left}.  Otherwise,
    %   if~$\blocked(p',\alpha,m)=1$, then~$d_1$ and~$d_2$ are instances of
    %   axiom~\eqref{def:orig:inner:or-attr-right}, and as such,
    %   $\alpha_1=\alpha_2=\alpha$, and as~$\eval$ and~$\clear$ are total
    %   functions, $m_1=m_2=m[v/\eval(a)]$ and~$p_1=p_2=\ceu\clear(p')$.
      %%
    \item$p'=\ceu\Break$.  Then~$d_1$ and~$d_2$ are instances of
      axiom~\eqref{def:orig:inner:or-break-left}, and as such,
      $\alpha_1=\alpha_2=\alpha$, $m_1=m_2=m$, and as~$\clear$ is a total
      function, $p_1=p_2=\ceu{\clear(p'');\Break}$.
      %%
    \item $p'\ne\ceu{\Skip,\Break}$ and~$p''=\ceu\Break$.
      If~$\blocked(p',\alpha,m)=0$, this case becomes
      \Cref{thm:orig:det-inner:or-left}.  Otherwise,
      if~$\blocked(p',\alpha,m)=1$, then~$d_1$ and~$d_2$ are instances of
      axiom~\eqref{def:orig:inner:or-break-right}, and as such,
      $\alpha_1=\alpha_2=\alpha$, $m_1=m_2=m$, and as~$\clear$ is a total
      function, $p_1=p_2=\ceu{\clear(p');\Break}$.
      %%
    \item\label{thm:orig:det-inner:or-left}
      $p',p''\ne\ceu{\Skip,\Break}$.
      If~$\blocked(p',\alpha,m)=0$, $d_1$~and~$d_2$ are instances
      of~\eqref{def:orig:inner:or-left}.  Thus there are derivations~$d_1'$
      and~$d_2'$ such that
      \[
        d_1'\Vdash\<p',\alpha,m>\step{n}\<p_1',\alpha_1,m_1>
        \quad\text{and}\quad
        d_2'\Vdash\<p',\alpha,m>\step{n}\<p_2',\alpha_2,m_2>,
      \]
      for some~$p_1'$, $p_2'\in{P}$.  Since~$d_1'\prec{d_1}$
      and~$d_2'\prec{d_2}$, by induction hypothesis, $\alpha_1=\alpha_2$,
      $m_1=m_2$, and~$p_1'=p_2'$, which implies
      \[
        p_1=(\ceu{p_1'\Or{p''}})=(\ceu{p_2'\Or{p''}})=p_2.
      \]

      If, however, $\blocked(p',\alpha,m)=1$, then~$d_1$ and~$d_2$ are
      instances of~~\eqref{def:orig:inner:or-right}.  Thus there are
      derivations~$d_1''$ and~$d_2''$ such that
      \[
        d_1''\Vdash\<p'',\alpha,m>\step{n}\<p_1'',\alpha_1,m_1>
        \quad\text{and}\quad
        d_2''\Vdash\<p'',\alpha,m>\step{n}\<p_2'',\alpha_2,m_2>,
      \]
      for some~$p_1''$, $p_2''\in{P}$.  Since~$d_1''\prec{d_1}$
      and~$d_2''\prec{d_2}$, by induction hypothesis, $\alpha_1=\alpha_2$,
      $m_1=m_2$, and~$p_1''=p_2''$, which implies
      \[
        p_1=(\ceu{p'\Or{p_1''}})=(\ceu{p'\Or{p_2''}})=p_2.\qedhere
      \]
    \end{case}
  \end{case}
\end{proof}
